\capitulo{1}{Introducción}


La enfermedad de Párkinson es el segundo trastorno neurodegenerativo más común, cuyo efecto fisiológico principal es la pérdida progresiva de neuronas dopaminérgicas en la sustancia negra del cerebro. Los síntomas más comunes causados por esta disminución de dopamina son temblores, rigidez muscular y bradicinesia (lentitud en los movimientos). Estos síntomas motores en ocasiones se traducen en bloqueos de la marcha, los cuales pueden alterarla, suponer un cese total de la misma u ocasionar caídas. La monitorización continuada de los síntomas motores del Párkinson es crucial para la correcta adaptación del tratamiento a la evolución de la enfermedad por parte de los profesionales sanitarios, garantizando una mejor calidad de vida de los pacientes.

Este Trabajo de Fin de Grado (TFG) se centra en la monitorización de los bloqueos de la marcha que desembocan en un cese total de la misma. Esto se logra mediante el uso de un dispositivo hardware con componentes Arduino que registra dichos bloqueos, proporcionando también una ayuda visual para su superación, y una aplicación web de sencillo manejo destinada a la utilización por parte del paciente y el profesional sanitario. 

Esta solución tecnológica evoluciona a partir del diseño de hardware Arduino y aplicación web propuestos en los TFGs \cite{Gonzalez2023} \cite{Martos2024}, agregando mejoras en el hardware y software para una ampliación de sus funciones. Se incluye en la aplicación web la posibilidad de generar gráficas e informes, así como personalizar el tiempo de cese de la marcha necesario para que el sensor detecte un bloqueo muscular con congelamiento de la marcha mediante una sencilla prueba. Se incluyen también 2 diarios del paciente (de tomas de medicaciones y fluctuaciones motoras, los cuales al contrastarse con las actividades registradas proveen información sobre el efecto de las medicaciones en los sintomas motores del paciente. Se agrega una respuesta a la detección de bloqueos, proyectando una línea láser en el suelo para ayudar al paciente a recuperar la marcha de forma más sencilla.

En el presente documento se presenta la memoria del proyecto, comenzando por una descripción de los objetivos propuestos, una serie de conceptos teóricos relevantes para comprender el contexto de la presente solución y las tecnologías actuales en su misma línea, un apartado que describe la metodología utilizada y una descripción de los resultados obtenidos, conclusiones y líneas futuras para la mejora del proyecto.

Se incluirán adicionalmente los siguientes anexos: Anexo A, referido a la planificación temporal del desarrollo del proyecto; Anexo B, detallando un manual de usuario; Anexo C, constituyendo un manual del programador; Anexo D, con especificaciones sobre los datos con los que se trabaja durante el proyecto; Anexo E, detallando el diseño del hardware y software mediante planos y diagramas; Anexo F, sobre la especificación de requisitos; Anexo G, explicando un desarrollo experimental y anexo H, representando un anexo de sostenibilidad curricular.

Los documentos mencionados, así como los archivos de código y demás archivos relevantes se encuentran organizados en carpetas dentro del repositorio \href{https://github.com/cmg1015/TFG_Seguimiento_Parkinson}{GitHub}, en el link '\url{https://github.com/cmg1015/TFG_Seguimiento_Parkinson}'.




