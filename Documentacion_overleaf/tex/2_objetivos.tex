\capitulo{2}{Objetivos}
Este proyecto busca desarrollar una solución tecnológica hadware/software para el monitoreo de actividades y fluctuaciones motoras en pacientes con párkinson, consistente en un cinturón con una placa y otros módulos Arduino conectado a una tobillera con un sensor giroscopio/acelerómetro y una aplicación web. \\ \\
Mediante la comunicación bidireccional constante entre el hardware Arduino y el backend de la aplicación web se consigue registrar en tiempo real las actividades realizadas con su correspondiente fecha y hora, así como la reacción en tiempo real a los bloqueos detectados, contribuyendo a la reanudación de la marcha mediante la proyección de una línea láser en el suelo.\\ \\
La utilización regular de la aplicación web y el dispositivo permite obtener y analizar información relativa a la toma de medicaciones, fluctuaciones ON/OFF detectadas por el paciente y las actividades realizadas. El análisis de este conjunto de información supone una herramienta útil para la evaluación del progreso de la enfermedad y del tratamiento actual.
\subsection{Objetivos principales}
\begin{enumerate}
    \item Diseñar y construir un prototipo hardware consistente en un cinturón y una tobillera, que resulte cómodo y de bajo coste y que reaccione al detectar una congelación de la marcha mediante una proyección de línea láser en el suelo.
    \item Implementar funcionalidades software para la personalización del funcionamiento del hardware desde la aplicación web y el registro de tomas de medicación y sintomatología diario, el cual será analizado para obtener datos relevantes sobre la evolución de la enfermedad y la efectividad del tratamiento.
\end{enumerate}

\subsection{Objetivos de hardware}
\begin{enumerate}
    \item Diseñar un prototipo que, contando con todos los módulos descritos en el trabajo de fin de grado \cite{Martos2024}, resulte flexible y pueda colocarse en un cinturón con compartimentos para una mayor comodidad del usuario. 
    \item Añadir un componente de módulo láser, que proyecte una línea en el suelo al detectar un bloqueo muscular con cese de la marcha para ayudar al paciente a reanudarla.
    \item Agregar un componente RTC, con el objetivo de registrar la fecha y hora de realización de las actividades.
\end{enumerate}

\subsection{Objetivos de desarrollo software}
\subsubsection{Funciones de la aplicación web}
\begin{enumerate}
    \item Agregar una prueba de personalización, que permita registrar una actividad con ciertas particularidades: se omitirá la detección de bloqueos y se permitirá la finalización y guardado de resultados tan sólo si se han superado los 10 pasos. Esta prueba se utilizará para registrar el tiempo medio que tarda el paciente en dar un paso, y a partir de ello determinar el número de segundos en reposo que debe detectar el sensor para detectar un bloqueo muscular con cese de la marcha. 
    Estas pruebas se guardarán en una tabla separada de la base de datos y el número de segundos determinado será enviado al hardware, que funcionará acorde al mismo desde ese momento en adelante.
    \item Como nueva funcionalidad del tipo de usuario 'profesional', se podrá agregar una pauta de medicación determinada a cada paciente.
    \item Como nueva funcionalidad del tipo de usuario 'paciente' se incluye la posibilidad de registrar un diario, en el que se registrarán las tomas de medicación y los estados ON/OFF del paciente. Al resultar normalmente repetitivas las tomas de medicación diarias, se establecerán como respuestas predeterminadas al formulario la pauta de medicación que el profesional haya asignado al paciente.
\end{enumerate}
\subsubsection{Funciones del código Arduino}
\begin{enumerate}
    \item Manejar de forma adecuada el encendido y apagado del módulo láser tras la detección de cese de la marcha.
    \item Manejo del módulo RTC, almacenando junto con la información de la actividad (bloqueos, tiempo, pasos...) la fecha y hora en la que ha sido realizada, y enviando dichos valores de forma adecuada a los archivos de código correspondientes al backend de la aplicación web.
    \item Flexibilizar la detección de bloqueos, cambiando el número fijo de segundos de reposo del sensor a partir de los cuales se detecta un cese de la marcha (5 segundos) por un número variable entre 1 y 20 segundos, en función de la información recibida desde la app.
\end{enumerate}
%Objetivos principales del trabajo realizado.


%Este apartado explica de forma precisa y concisa cuales son los objetivos que se persiguen con la realización del proyecto. Se puede distinguir entre:
%\begin{enumerate}
 %   \item Los objetivos marcados por los requisitos del software/hardware/análisis a desarrollar.
  %  \item Los objetivos de carácter técnico, relativos a la calidad de los resultados, velocidad de ejecución, fiabilidad o similares.
   % \item Los objetivos de aprendizaje, relativos a aprender técnicas o herramientas de interés. 
%\end{enumerate}








