\apendice{Anexo de sostenibilización curricular}

El grado en Ingeniería de la Salud, a partir del cual se presenta el presente Trabajo de Fin de Grado está alineado de manera natural con el respeto a los derechos fundamentales de las personas, ya que el desarrollo de software y/o hardware para complementar el tratamiento sanitario garantiza el acceso al sistema sanitario de personas en áreas rurales, con pocos recursos económicos o situados en zonas con baja calidad en la atención sanitaria. 

Este respeto al derecho a la salud de todas las personas se entrelaza también con el respeto a la igualdad de trato y no discriminación, promoviendo la inclusión social de personas que padecen enfermedades incapacitantes o discapacidades mediante el diseño de soluciones personalizadas que adapten el acceso a su entorno a las necesidades de estas personas.

El presente proyecto pretende aplicar estos principios proporcionando una ayuda a personas con Párkinson y los profesionales que los atienden, manteniendo al mismo tiempo un compromiso con la sostenibilidad del producto mediante la cuidadosa elección de los componentes hardware. A continuación se detalla la forma de cumplimiento de estos derechos y obligaciones:
\begin{itemize}
    \item Cumplimiento del respeto a los derechos humanos o derechos fundamentales: El presente proyecto se alinea con el derecho a la salud, aportando una solución accesible y barata para la monitorización de síntomas motores en pacientes con párkinson. Garantizar esta monitorización permite hacer la actuación del profesional sanitario más efectiva, ya que le proporciona información muy relevante para la toma de decisiones médicas.
    \item Cumplimiento del respeto a la igualdad de género, igualdad de trato y no discriminación: Esta solución tecnológica pretende garantizar un simple manejo de la aplicación web, haciéndola accesible a todas las personas independientemente de su edad, género, formación... Adicionalmente, promueve la inclusión social de personas con Párkinson, al ser una herramienta que contribuye a recibir un mejor trato, reduciendo los síntomas motores que a menudo provocan el aislameniento de estos pacientes.
    \item Compromiso con la sostenibilidad y concienciación con el cambio climático: El presente proyecto utiliza una serie de productos con una larga vida útil, incluyendo una fuente de alimentación recargable. Esto reduce la cantidad de resíduos generados durante el uso del dispositivo. Adicionalmente, durante la realización del prototipo componentes como la pantalla LCD, el sensor MPU-6050 y la placa Arduino no se soldaron de forma directa a los cables del circuito, permitiendo la reutilización de los mismos cuando el dispositivo deje de ser útil o quiera rediseñarse.
\end{itemize}
